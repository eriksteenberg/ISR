\documentclass[danish,a4paper,twocolumn]{article}
\usepackage[margin=2cm]{geometry}
\usepackage[utf8]{inputenc}
\usepackage[T1]{fontenc}
\usepackage{babel}
\usepackage{color}
\usepackage{titlesec}
%\usepackage[sc]{mathpazo}

\newcommand{\foreningen}{Institut for Fysik \& Astronomi StudenterRepræsentation}

% Smarte referencer
\usepackage{smartref}
\addtoreflist{section}
\addtoreflist{subsection}
\newcommand{\longref}[1]{\sectionref{#1}, \subsectionref{#1}}

% Typografi
\setlength\parindent{0pt}
\renewcommand{\thesection}{\S\arabic{section}}
\renewcommand{\thesubsection}{Stk.~\arabic{subsection}}
\titlespacing*{\section}{0pt}{10pt}{5pt}
\titleformat{\section}{\normalfont\large\bfseries}{\thesection}{1em}{}
\titlespacing*{\subsection}{0pt}{9pt}{0pt}
\titleformat{\subsection}{\normalfont\bfseries}{\thesubsection}{1em}{}



% Ordeling
\hyphenation{uni-ver-si-tet fler-tal}

\begin{document}
\title{\vspace{-2ex}Vedtægt for \foreningen\vspace{-5ex}}
\date{\today}
\maketitle

\section{Formål og tilhørsforhold}
\subsection{}Foreningens navn er \foreningen, (ISR).
\subsection{}\label{stk:sigma} 
ISR er en underforening af SIGMA
\subsection{}
Medlemmer af ISR er ikke automatisk medlem af SIGMA og kan derfor ikke blive stillet de samme krav som der kan stilles af aktive SIGMA medlemmer.
\subsection{}
Som underforening af SIGMA forpligtiger tovholderen i ISR sig på at sende referater til SIGMA og regelmæssigt underrette formanden for SIGMA om status i ISR.
\subsection{}ISR har hjemsted på Institut for Fysik \& Astronomi.
\subsection{}ISR består af studerende ved fysik og astronomi på Aarhus universitet, som repræsenterer de studerende i samtale med instituttets ledelse.
\section{Valg af medlemmer}
Hver årgang på IFA har krav på to medlemmer i ISR, en repræsentant og en suppleant. Disse vælges internt på årgangen og efter egen metode.
\subsection{}
Kan der på en årgang ikke opnås enighed om en valgmetode, eller vurderes der fra siddende repræsentation at være overvejende utilfredshed med valget, kan to medlemmer fra den siddende studenterrepræsentation overvære og vejlede årgangens valg.
\subsection{}
!!!!Skal fikses!!!\\
Nyligt valgte medlemmer af ISR skal senest syv dage inden forårssemesterets
læseferie være valgt og indberettet til den siddende studenterrepræsentation. Denne indberetning foregår som udgangspunkt på et åbent møde som ISR indkalder til før tidsfristen. Som udgangspunkt holdes et introducerende møde for de nyligt valgte medlemmer mellem dette tidspunkt og læseferiens begyndelse. 

\subsection{}
Nyligt valgte medlemmer af ISR tiltræder som siddende repræsentation dagen efter sidste dag i eksamensperioden i forårssemesteret.

\subsection{}
Studerende på første år skal have valgt  deres repræsentant til ISR inden  efterårssemesterets læseferie. Dette skal arrangeres af ISR. 
%\subsection{}
%Valg af førsteårsstuderende til ISR ved et sådan arrangement skal finde sted ved anonymt valg.
%\subsection{}
%Der holdes som udgangspunkt ikke et introducerende møde i denne sammenhæng.Første årgangs vælges på lige fod med de andre årgange igen nye (eller %eventueltsamme) medlemmer inden forårssemesterets læseferie som specificeret herover.
\subsection{}
Et medlem fra en årgang kan kun udskiftes i tilfælde af, at vedkommende
frivilligt trækker sig, eller at mindst halvdelen af den pågældende årgang stemmer for et mistillidsvotum. Det er op til ISR at vurdere om en sådan afstemning er fundet sted under retfærdige vilkår.
\subsection{}
Hvis man er forsinket eller har et ikke-standard studieforløb, således at man ikke følges med sin oprindelige årgang, kan man vælge at stille op for den årgang man føler at man bedst repræsenterer. Man kan kun stille op for en årgang ved en given valgperiode.
\subsection{}
Man kan ikke stille op til ISR, hvis man regner med at dimmitere mere end 1
måned inden perioden, der vælges til, slutter.
%\section{SIGMA}'
\subsection{}
Repræsentantens opgave er at sidde med til interne og åbne møder i ISR, samt
ledelsesmøder med IFA. Suppleantens opgave er at sidde med til interne og åbne møder i ISR, samt hvis en repræsentant er forhindret at deltage til ledelsesmøder med IFA.
\section{Møder}
\subsection{}
ISR holder som udgangspunkt minimum fire åbne møder om året, et umiddelbart før hvert møde med ledelsen, samt interne møder for at diskutere forslag og problemstillinger på de studerendes vegne.

\subsection{}
Fire gange årligt mødes fem medlemmer (fire i efterårssemesteret, da førsteårgang endnu ikke er repræsenteret), en fra hver årgang, med IFAs ledelse, herunder institutlederen. Er både repræsentanten og suppleanten fra en årgang forhindret, bør en suppleant fra en anden årgang træde til i deres sted.

\subsection{}
Interne møder afholdes efter vurderet behov, samt minimum i samme uge inden
et ledelsesmøde. Til interne møder deltager som udgangspunkt kun medlemmer
adf ISR og SIGMA og specifikt inviterede tilknyttet relevante studenterorganer (Dynastiet,TÅGEKAMMERET, osv.). 
\subsection{}
Som udgangspunkt påbegyndes mødet med valg af ordstyrer og referent. Er
der bred uenighed om dette blandt de deltagende ved mødet, kan tovholderen
afholde en åben afstemning om posterne. Efterfølgende følges der op på tidligere møder. 
\subsection{}Kræver et emne fra et tidligere møde mere fokus end det der svarer til en kort opsummering, tages dette emne op igen som sit eget punkt på mødets dagsorden.
\section{Åbne møder}
 \subsection{}
Åbne møder afholdes efter vurderet behov, ofte i forbindelse med et eller 
flere store emner, hvor de studerende bør bringes i direkte dialog med repræsentationen. Alle interesserede  deltage i et åbent møde. Til åbne møder er referent og ordstyrer valgt på forhånd efter reglerne specificeret i "Interne møder". Dette vil typisk finde sted som det sidste punkt på dagsordenen på det foregående interne møde.
\subsection{}
Til åbne møder er alle afstemninger anonyme. Dette gøres for at beskytte deltagende ved mødet i tilfælde af kontroversielle afstemninger, og for at beskytte studenterrepræsentationen fra, at skulle beslutte hvilke afstemninger er kontroversielle og hvilke ikke er. %Normen omstændigheder for afstemningen besluttes i de enkelte tilfælde af studenterrepræsentationen på foregående interne møde.
\section{Interne valg}
\subsection{}
Internt i repræsentationen forefindes en tovholder. Dennes opgave er, at holde rede i repræsentationens aktuelle handlingsplaner, samt bringe de andre medlemmer til medansvar for, at omtalte handlingsplaner gennemføres i god tid. Derudovers sørger tovholderen for, at møder bliver afholdt som specificeret i retningslinjerne. Dermed menes ikke at tovholderen kan sætte møderne som ønsket, men blot at denne har til ansvar at et møde planlægges på tværs af repræsentationen inden for de rette tidsrum.
\subsection{}Tovholderen forpligter sig på at informere formanden for SIGMA om status i ISR.

\subsection{}
Tovholderen vælges ved åbent valg blandt medlemmerne ved første interne
møde. Kun fremmødte medlemmer kan stemme på og opstilles som tovholder.
\subsection{}
Er ISR ikke fungerende, kan SIGMA indsætte en fungerende tovholder, indtil en ny kan vælges internt i ISR.
\subsection{}
Ved utilfredshed ved tovholderens arbejde kan et nyt valg påbegyndes hvis minimum to tredjedele af repræsentationen stemmer for et mistillidsvotum.
\subsection{}
På nær under anden aftale forventes tovholderen at planlægge ledelsesmøder
med IFA.
\section{Ændring af vedtægter}
\subsection{}
For at ændre ISRs vedtægter skal et ændringsforslag først have mere end $2/3$ af alle siddende medlemmers stemme. Derefter skal ændringsforslaget godkendes på det efterfølgende SIGMA møde. 

\section{ Eksterne relationer}
ISR repræsenterer de studerende på fysik og astronomi på Aarhus universitet i forhold til problemstillinger og forslag omhandlende IFAs ledelse. ISR er som hovedfunktion et kommunikationsorgan, som skal videregive information fra ledelsen til de studerende, og feedback og forslag fra de studerende til ledelsen.
\subsection{}
Som underforening af SIGMA har ISR ingen politisk dagsorden. Dette betyder, at der internt ikke vil blive taget stilling til, hvorvidt en problemstilling fremstillet af en studerende bør tages op med ledelsen eller ej, baseret på medlemmer af repræsentationens politiske holdninger. 
\subsection{}
Problemstillinger fremstillet af studerende til ISR kan bedømmes at være uegnede til at fremstille for ledelsen, men i et sådan tilfælde vil en anden
passende instans foreslås kontaktet af den eller de pågældende studerende.
\subsection{}
Som udgangspunkt blander ISR sig ikke i den enkelte undervisers undervisningsmetode, men henviser derimod studerende til SIGMA, som bedre kan tage hånd om et sådant problem. I tilfælde hvor der er tale om mere generelle tiltag til et enkelt eller  flere kurser, hvor ISR vurderer, at forslag med fordel kan præsenteres for ledelsen, er det ISRs opgave at tage hånd om dette. ISR blander sig ikke i kurser som er udbudt uden for IFA.

\section{Udtrædelse af SIGMA}

\subsection{}
Udtrædelse af ISR fra SIGMA skal godkendes på et møde i ISR, et studentermøde på Institut for Fysik & Astronomi og et møde i SIGMA med 2/3 flertal i alle tilfælde.
\subsection{}
Til et studentermøde om udtrædelse af ISR, må kun dette emne behandles. Kun fysik- og astronomistuderende vil have stemmeret.
\\
\\
Fungerende Tovholder:\\ Erik Holm Steenberg 20-02-2023
\end{document}

%%% Local Variables:
%%% mode: latex
%%% TeX-master: t
%%% End:
